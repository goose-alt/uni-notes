\chapter{Indledning}
\section{Baggrund for rapport}
Denne rapport er skrevet i forbindelse med projektarbejdeforløbet i faget \emph{Projektarbejde og Kommunikation} på 1. semester af bacheloruddannelsen Softwareudvikling på IT-Universitetet i København. Rapporten er udarbejdet med henblik på at løse et problem, givet af virksomheden \emph{SC-Tronics} (SC-T) i forbindelse med lancering af deres nye produkt \emph{saWux}. Udfordringen går på at forbedre datavisualiseringen for \emph{saWux'} produkter, således at den viste data skaber mest mulig værdi for kunden.\\
Rapporten danner grundlag for at assistere virksomheden i at omstrukturere deres allerede eksisterende app med henblik på bedre visualisering af data.

\section{Problemfelt}
Flere virksomheder og husstande ønsker øget kontrol og forståelse for deres energiforbrug \citep{klimaindsats} — for både at spare penge og blive mere bæredygtige. Statistiske og grafiske værktøjer er fordelagtige i forbindelse med at øge netop kontrollen, og de skaber et bedre overblik for brugerne af deres energiforbrug \citep{herrmann}. I store virksomheder anvendes allerede omfattende værktøjer, komplekse løsninger og store pengesummer til at optimere energiforbruget. Ifølge det danske udviklings- og energistyringsfirma SC-T eksisterer der et hul i markedet for løsninger til private husstande og til de små og mellemstore virksomheder \citep{sawux}.

\section{Problemstilling}
SC-Tronics har udarbejdet løsningen \emph{saWux} til energistyring og dataindsamling, som iværksættervirksomheden i skrivende stund lancerer i de kommende måneder. En udfordring er, at data ikke fremstår tydeligt for brugeren af den tilknyttede \emph{saWux}-app. Observation og visualisering af data spiller en central rolle i, at målgruppen har lyst til at investere i og bruge SC-T produkter. Når kunden investerer i \emph{saWux}, regner man med besparelse på 10-20 procent i energiforbrug om året \citep{sawux}, hvilket skal gøres klart og forståeligt for brugeren. Appen muliggør at kontinuerligt observere elforbruget, men kommunikationen og formidling af eventuelle besparelser og data fra virksomhedens øvrige enheder er enten ikke tilstedeværende eller uklart.

\section{Problemformulering}
Hvordan kommunikerer og visualiserer SC-T elforbrug til de private husstande med særligt henblik på familier med hjemmeboende børn, således at målgruppen motiveres til strømbesparende adfærd og aktiv, miljørigtig handling?