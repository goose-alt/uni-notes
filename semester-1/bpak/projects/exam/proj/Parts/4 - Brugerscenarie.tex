\chapter{Brugerscenarie}
For at uddybe funktionaliteten af vores produkt opstilles et brugerscenarie, der inddrager en far og hans datter.\\

\textbf{Bruger 1:}\\
\textbf{Navn:} Karsten\\
\textbf{Køn og Alder:} Mand, 45 år\\
\textbf{Rolle:} Familiefar og tech-entusiast\\
\textbf{Brug:} Skabe overblik og kontrol over energiforbruget i forskellige dele af huset\\[0.5ex]

\textbf{Bruger 2:}\\
\textbf{Navn:} Lotte\\
\textbf{Køn og Alder:} Pige, 13 år\\
\textbf{Rolle:} Datter til Karsten; går i 5. klasse\\
\textbf{Brug:} Konkurrence mellem familiemedlemmer\\[0.5ex]

\begin{enumerate}
    \item Karsten kommer hjem fra arbejde og åbner \emph{saWux Visualizer} på sin smartphone.
    \item Han scroller igennem sit \emph{dashboard} og ser, at der er blevet brugt usædvanligt meget strøm i huset i løbet af dagen.
    \item Karsten undersøger problemet og opdager hurtigt, at fjernsynet har stået tændt, uden at nogen brugte det. 
    \item For at få bedre overblik over strømforbruget fra fjernsynet opretter han et nyt \emph{card} på sit \emph{dashboard}, der ser strømforbruget i stuen, hvor fjernsynet befinder sig. Karsten vælger at visualisere strømforburget via en graf.

    \item Lotte kommer hjem fra skole og sætter sig på sit værelse, hvor hun tænder for sit fjernsyn.
    \item Hun åbner \emph{saWux Visualizer}, der allerede er inde på \emph{Family Mode}. Her kan hun se familiens igangværende \emph{First to 25}-konkurrence. Hun observerer under \emph{Rankings}, at hun har brugt flest kWh og derfor er ved at tabe konkurrencen. 
    \item Hun beslutter sig for at slukke for fjernsynet, stikkontakten ligeledes, for at spare på strømmen.
    \item Lotte ser på sit \emph{dashboard}, at dette får strømforbruget på hendes værelse til at falde.
    \item Samtidig sidder Karsten i familiens stue og scroller igennem \emph{Inspiration Mode} i \emph{saWux Visualizer}.
    \item Her opdager han en artikel med gode råd til hans \emph{saWux} produkter.
\end{enumerate}

