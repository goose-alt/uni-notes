%!TEX root = ../[BDSA'21] - Exam Answers.tex

\pgfmathsetmacro\totalpoints{\totalpoints + 5}

\noindent You need to store \textit{Knights of the Round Table} in a table.\\
\noindent You will also have a table of available seats at the \textit{Round Table}.\\
\noindent There are 24 seats at the table, however, the royal carpenter is said to be working on a bigger table\ldots\\
\noindent A knight can be in exactly one position at the table at any given time but does not have to be seated.\\
\noindent A knight will have the following attributes:
\begin{itemize}
	\item Id : globally unique value (there can only ever be the one Lancelot\ldots)
	\item Title : size up to 20
	\item Name : mandatory, length 50
\end{itemize}
\noindent A table seat will have the following attributes:
\begin{itemize}
	\item Position : (number going up clockwise starting from 0 (King Arthur))
\end{itemize}
\noindent You decide to store the knights and table in relational database tables.\\
\noindent You must to decide how to model where a \textit{knight} is \textit{seated}.

\vskip 15pt
\begin{enumerate}[a]
    \item \point{3} Complete a table for each table:

\noindent Table name:\ %
			\ifdefined\questionElevenAnswerAfirstTableName
			  \newline\answer\questionElevenAnswerAfirstTableName
			\else
			  \dodotrule{.95}
			\fi

\begin{center}
	\begin{tabular}{ p{0.8cm}  p{3.5cm}  p{3.5cm}  c }
	\toprule
		\multicolumn{1}{c}{\textbf{Key?}} &
		\multicolumn{1}{c}{\textbf{Column Name}} &
		\multicolumn{1}{c}{\textbf{Data Type}} &
		\multicolumn{1}{c}{\textbf{Allow Nulls}} \\
	\specialrule{.1em}{.05em}{.05em}
		\questionElevenAnswerAfirstTableRowOneColumnKey &
		\questionElevenAnswerAfirstTableRowOneColumnName &
		\questionElevenAnswerAfirstTableRowOneColumnType &
		\questionElevenAnswerAfirstTableRowOneColumnAllowNull \\
	\midrule
		\questionElevenAnswerAfirstTableRowTwoColumnKey &
		\questionElevenAnswerAfirstTableRowTwoColumnName &
		\questionElevenAnswerAfirstTableRowTwoColumnType &
		\questionElevenAnswerAfirstTableRowTwoColumnAllowNull \\
	\midrule
		\questionElevenAnswerAfirstTableRowThreeColumnKey &
		\questionElevenAnswerAfirstTableRowThreeColumnName &
		\questionElevenAnswerAfirstTableRowThreeColumnType &
		\questionElevenAnswerAfirstTableRowThreeColumnAllowNull \\
	\midrule
		\questionElevenAnswerAfirstTableRowFourColumnKey &
		\questionElevenAnswerAfirstTableRowFourColumnName &
		\questionElevenAnswerAfirstTableRowFourColumnType &
		\questionElevenAnswerAfirstTableRowFourColumnAllowNull \\
	\midrule
		\questionElevenAnswerAfirstTableRowFiveColumnKey &
		\questionElevenAnswerAfirstTableRowFiveColumnName &
		\questionElevenAnswerAfirstTableRowFiveColumnType &
		\questionElevenAnswerAfirstTableRowFiveColumnAllowNull \\
	\midrule
		\questionElevenAnswerAfirstTableRowSixColumnKey &
		\questionElevenAnswerAfirstTableRowSixColumnName &
		\questionElevenAnswerAfirstTableRowSixColumnType &
		\questionElevenAnswerAfirstTableRowSixColumnAllowNull \\
	\bottomrule
\end{tabular}
\end{center}





\newpage
\noindent Table name:\ %
			\ifdefined\questionElevenAnswerAsecondTableName
			  \newline\answer\questionElevenAnswerAsecondTableName
			\else
			  \dodotrule{.95}
			\fi

\begin{center}
	\begin{tabular}{ p{0.8cm}  p{3.5cm}  p{3.5cm}  c }
	\toprule
		\multicolumn{1}{c}{\textbf{Key?}} &
		\multicolumn{1}{c}{\textbf{Column Name}} &
		\multicolumn{1}{c}{\textbf{Data Type}} &
		\multicolumn{1}{c}{\textbf{Allow Nulls}} \\
	\specialrule{.1em}{.05em}{.05em}
		\questionElevenAnswerAsecondTableRowOneColumnKey &
		\questionElevenAnswerAsecondTableRowOneColumnName &
		\questionElevenAnswerAsecondTableRowOneColumnType &
		\questionElevenAnswerAsecondTableRowOneColumnAllowNull \\
	\midrule
		\questionElevenAnswerAsecondTableRowTwoColumnKey &
		\questionElevenAnswerAsecondTableRowTwoColumnName &
		\questionElevenAnswerAsecondTableRowTwoColumnType &
		\questionElevenAnswerAsecondTableRowTwoColumnAllowNull \\
	\midrule
		\questionElevenAnswerAsecondTableRowThreeColumnKey &
		\questionElevenAnswerAsecondTableRowThreeColumnName &
		\questionElevenAnswerAsecondTableRowThreeColumnType &
		\questionElevenAnswerAsecondTableRowThreeColumnAllowNull \\
	\midrule
		\questionElevenAnswerAsecondTableRowFourColumnKey &
		\questionElevenAnswerAsecondTableRowFourColumnName &
		\questionElevenAnswerAsecondTableRowFourColumnType &
		\questionElevenAnswerAsecondTableRowFourColumnAllowNull \\
	\midrule
		\questionElevenAnswerAsecondTableRowFiveColumnKey &
		\questionElevenAnswerAsecondTableRowFiveColumnName &
		\questionElevenAnswerAsecondTableRowFiveColumnType &
		\questionElevenAnswerAsecondTableRowFiveColumnAllowNull \\
	\midrule
		\questionElevenAnswerAsecondTableRowSixColumnKey &
		\questionElevenAnswerAsecondTableRowSixColumnName &
		\questionElevenAnswerAsecondTableRowSixColumnType &
		\questionElevenAnswerAsecondTableRowSixColumnAllowNull \\
	\bottomrule
\end{tabular}
\end{center}





\vskip 15pt
    \item \point{2} Below, list your chosen primary -- should already be marked in the table -- and foreign key constraints.
		\ifdefined\questionElevenAnswerB
		  \newline\answer\questionElevenAnswerB
		\else
		  \\\dodotrule{.95}\dodotrule{.95}\dodotrule{.95}\dodotrule{.95}\dodotrule{.95}
		\fi

\end{enumerate}












