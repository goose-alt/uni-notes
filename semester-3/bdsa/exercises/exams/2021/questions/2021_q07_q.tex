%!TEX root = ../[BDSA'20] - Exam.tex

\pgfmathsetmacro\totalpoints{\totalpoints + 5}

Considering the \el{strategy pattern} and the \el{bridge pattern}, answer the following questions.

\begin{enumerate}[a.]

	\item \point{1} Explain the difference, if any, between the two.
		\ifdefined\questionSevenAnswerA
		  \newline\answer\questionSevenAnswerA
		\else
		  \\\dodotrule{.95}\dodotrule{.95}\dodotrule{.95}\dodotrule{.95}
		\fi





	\item \point{2} In the context of unit testing, which of the two would you consider as a helpful solution? What problem would you be solving?
		\ifdefined\questionSevenAnswerB
		  \newline\answer\questionSevenAnswerB
		\else
		  \\\dodotrule{.95}\dodotrule{.95}\dodotrule{.95}\dodotrule{.95}
		\fi





	\item \point{1} In the context of testing, what is the \underline{purpose} of using test doubles (i.e., dummy and fake objects, stubs, and mocks)?
		\ifdefined\questionSevenAnswerC
		  \newline\answer\questionSevenAnswerC
		\else
		  \\\dodotrule{.95}\dodotrule{.95}
		\fi





	\item \point{1} Focusing on integration testing, name 3 different integration testing strategies.
		\ifdefined\questionSevenAnswerD
		  \newline\answer\questionSevenAnswerD
		\else
		  \\\dodotrule{.95}\dodotrule{.95}\dodotrule{.95}\dodotrule{.95}\dodotrule{.95}\dodotrule{.95}
		\fi

\end{enumerate}
