%!TEX root = ../[BDSA'21] - Exam Answers.tex

\pgfmathsetmacro\totalpoints{\totalpoints + 8}

\point{8} Somehow, \textbf{Chiosco da Paolo} has managed to not only stay afloat in this period, but managed to grow in business due to its ability to deliver very custom orders to customers at their address.  However, Paolo relied on an old-fashion phone-based ordering process that clearly does not scale given that he is running Chiosco da Paolo as a one-man business to save on costs.  Given the successful demonstration of your solution to model the offerings in September, he turned again to you for technical support.  

You had the chance to have a chat with Paolo, and he told you all about the reasons he is not using other third party well-established systems.  Even though you tried to convince him that creating yet another ordering system is pointless, you did not manage to deter him from asking you to design the ordering system for his business.  You shrug and go ahead as it appears that what he needs is extremely straight forward.  During the conversation, Paolo mentioned that:
\begin{itemize}
	\item people should be able to `call' him (i.e., order throught the system) and place the order even if it is the first time they order at Chiosco da Paolo;
	\item regular customers should be able to register and later log-in to Chiosco da Paolo to benefit from discount campaigns.  Paolo was still unable to explain what or how these campaigns would look like;
	\item people should be able to order all the complex italian street food dishes that he described to you in September (see notes below) and as many of these as necessary.
	\end{itemize}

You also find your notes from September:
\begin{itemize}
	\item the place is meant to sell food. Paolo mentioned: pizza, calzone, toast, sandwitch, focacce, \ldots
	\item Paolo also mentioned that somehow he can do all the above with an assortment of ingredients, which can all be combined at the wimps of the customer. He mentioned: ham, cheese, muchrooms, gorgonzola, spinach, speck, nutella, jam, \ldots
	\item Paolo indicated that he wants the customer to be able to say that they want double, triple, \ldots amount of each ingredient
	\end{itemize}

Focusing only on the interactions between a customer and the online booking system that you are modeling for Paolo, your task at hand is to draw a sequence diagram containing enough details that would allow you to use the diagram while explaining to a hired developer the main interactions that the system will support to allow customers to place online orders.

\newpage
\fullline\vspace{-8pt}
\begin{center}{
	\scriptsize{\emph{Use this space between the horizontal lines to draw your final UML diagram.}}
	\includegraphics[width=\questionOneAnswerWidth\columnwidth]{\questionOneAnswer}
	}
\end{center}
\vfill
\fullline




