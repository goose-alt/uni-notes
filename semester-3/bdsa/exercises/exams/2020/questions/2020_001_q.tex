%!TEX root = ../[BDSA'21] - Exam Answers.tex




\pgfmathsetmacro\totalpoints{\totalpoints + 201}

Consider the following code:

\begin{lstlisting}
public class Genius
{
	public int Id { get; set; }
	public string Name { get; set; }
	public DateTime Birthday { get; set; }
	public string[] KnownFor { get; set; }
}

[Route("[controller]")]
[ApiController]
public class GeniusController : ControllerBase
{
	[HttpGet("{id}")]
	public async Task<IActionResult> Put(int id, [FromBody]Genius
	genius)
	{
		...
	}
}

public class GeniusRepository : IGeniusRepository
{
	// returns true if updated successfully;
	// false if something was wrong on the user side.
	public async Task<bool> Update(Genius genius)
	{
		...
	}
}
\end{lstlisting}

\begin{enumerate}[a]
    \item \point{3} Fill out the form to describe a HTTP request to update genius no. 1:
Grace Hopper born December 9, 1906 known for UNIVAC and COBOL.
        \begin{itemize}
			\item Api is hosted on https://genius.io
			\item Request should explicitly declare that it sends and expects JSON.
			\item You must use this token to authenticate: ey.  
		\end{itemize}t
		\ifdefined\questionOneAnswerA
		  \newline\answer\questionOneAnswerA
		\else
		  \\\dodotrule{.95}\dodotrule{.95}\dodotrule{.95}
		\fi

    \item \point{100} Complete the implementation for the ``hello world'' famous example. 
		\input{\questionOneAnswerB}

    \item \point{100} Try to write something, remember to try multiple lines as well as some environments like enumerate or itemize.  
		\ifdefined\questionOneAnswerC
		  \newline\answer\questionOneAnswerC
		\else
		  \\\dodotrule{.95}\dodotrule{.95}\dodotrule{.95}\dodotrule{.95}\dodotrule{.95}\dodotrule{.95}
		\fi

\end{enumerate}
