\subsection{Question 1}
What is the asymptotic running time of the following piece of code? (Choose the smallest correct estimate.)

\begin{lstlisting}[language=python]
i = 0
k = 0
while i < N:
    i = i+1
    if k > 0:
        j = 0 
        while j < N: 
            A[i] = A[i] + A[j] + k
            j = j + 1
    k = 1 - k;
\end{lstlisting}
The loop runs from 0 to n, on the first loop only it doesnt run the inner loop, but overall its quadratic\\[1ex]
\framebox[\textwidth]{Quadratic in $N$}


\subsection{Question 2}
How many stars are printed when I call $f(N)$?
\begin{lstlisting}[language=python]
def f(K):
    for i in range(K):  g(i)
def g(K):
    for i in range(K): print('*')
\end{lstlisting}
The closest is quadratic time\\[1ex]
\framebox[\textwidth]{$N^2$}


\subsection{Question 3}
How many stars are printed? (Choose the smallest correct estimate.)
\begin{lstlisting}[language=python]
for i in range(N):
    for j in range(i):
          print('*');
\end{lstlisting}
The closest is quadratic time\\[1ex]
\framebox[\textwidth]{$O(N^2)$}

\subsection{Question 4}
Which pair of functions satisfy $f(N)=O(g(N))$?\\
If we simplify then:\\[1ex]
\framebox[\textwidth]{$f(N)=(N+1)\cdot(N+1)\cdot(N+1)$ and $ g(N)=N^3$}


\subsection{Question 5}
What is the asymptotic running time of the following piece of code?
\begin{lstlisting}[language=python]
if N < 1000: 
    for i in range(N):
        for j in range(N):
            A[i] = j
  else :
    for i in range(N):
        A[i] = i
\end{lstlisting}
Since we are looking for the lowest then we assume $N>1000$:\\[1ex]
\framebox[\textwidth]{Linear in $N$}


\subsection{Question 6}
Consider the following sequence of operations
\begin{lstlisting}[language=python]
class A:
    def __init__(self):
        self.max = 1
        self.count = 0
    def f(self):
        self.count += 1
        if self.count == self.max:
            for i in range(self.max): print('*')
        self.max *= 2

a = A()
for i in range(N): 
    a.f()
\end{lstlisting}
How many stars are printed by the last call to a.f() in the sequence in the best case?\\
Since we are looking for the best case we can assume that the if statement isnt triggered, meaning we wouldnt print anything:\\[1ex]
\framebox[\textwidth]{Constant}


\subsection{Question 7}
Assume I have a function f taking a single integer argument (say, $K$) and which runs in amortised constant time, and logarithmic worst case time in $K$. What is the running time of 
\begin{lstlisting}[language=python]
for i in range(N): f(N)
\end{lstlisting}
We loop from 0 to $N$ and then run a method that can be either, on average, be constant or logarithmic in the worst case.
Since the function takes amortized constant time, aka on average calls take constant time, we can assume it to be constant. 
Which then is $N\cdot \text{constant} = N\cdot 1 = \text{linear}$\\[1ex]
\framebox[\textwidth]{Linear in $N$}


\subsection{Question 8}
Find a recurrence relation for the number of arithmetic operations (subtractions, additions, multiplications, divisions) performed by the following recursive method:
\begin{lstlisting}[language=python]
def f(N):
    if N > 1: return 2 * f(N - 1)
    else: return 3
\end{lstlisting}
For every recursion 2 arithmetic operations take place, therefore:\\[1ex]
\framebox[\textwidth]{$T(N) = T(N-1) + 2$}


\subsection{Question 9}
How many stars are printed?
\begin{lstlisting}[language=python]
i = N
while (i > 1):
    print ('*')
    i = i // 2
\end{lstlisting}
This is just the halfing operation\\[1ex]
\framebox[\textwidth]{$\log N$}


\subsection{Question 10}
Which pair of functions satisfy $f(N)=O(g(N))$?\\
A logarithm is basically just n+n+n a bunch of times, so one can simplify that way around\\[1ex]
\framebox[\textwidth]{$ f(N)=N+N+N$ and $g(N)=N\log N$}\


\subsection{Question 11}
Which pair of functions satisfy $f(N)\sim g(N)$?\\[1ex]
\framebox[\textwidth]{$ f(N)=N\log N$ and $g(N)=N\log N + 2N$}


\subsection{Question 12}
Consider the following piece of code:
\begin{lstlisting}[language=python]
N = len(s)
i = 0
while i < N:
    if s[i] == 0: return i
    i += 1
\end{lstlisting}
At the worst case the last element is the target, therefore 2 comparisons pr loop.\\[1ex]
\framebox[\textwidth]{The code uses $\sim 2N$ comparisons in the worst case.}


\subsection{Question 13}
How many stars are printed?
\begin{lstlisting}[language=python]
i = 1
while i < N:
    i = i+2
    stdio.write("*")
\end{lstlisting}
At every loop it skips 2 therefore it must be half\\[1ex]
\framebox[\textwidth]{$\sim \frac{N}{2}$}


\subsection{Question 14}
Which pair of functions satisfy $f(N)\sim g(N)$?\\
Just remove the quantifiers\\[1ex]
\framebox[\textwidth]{$f(N)=2\sqrt{N} + N$ and $g(N)=\sqrt{N}+N$}